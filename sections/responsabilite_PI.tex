% -------------------------------
% 5. Gestion, Responsabilité et Propriété Intellectuelle
% -------------------------------

\setlongtables
\renewcommand{\LTcapwidth}{\textwidth} % largeur des captions
\setcounter{LTchunksize}{0} % empêche la répétition automatique

% Supprimer le trait en haut seulement pour la page 8
%\clearpage
%\thispagestyle{nohead} 

\section{Gestion, Responsabilité et Propriété Intellectuelle}

\subsection{Propriété intellectuelle et souveraineté}

\subsubsection{Propriété du modèle de décision}
Dans un contexte académique, le modèle de décision développé dans le cadre de ce projet \textbf{NavetteAuto} constitue une création intellectuelle au sens de la législation sur les logiciels.

Dans une université, la règle générale est la suivante :

\begin{itemize}
    \item Les droits patrimoniaux appartiennent à l’UQO, conformément aux politiques internes sur la propriété intellectuelle des travaux réalisés dans le cadre de projets pédagogiques ou de recherche.
    \item Les étudiants développeurs conservent des droits moraux (paternité, intégrité de l’œuvre), mais n’ont pas la maîtrise de l’exploitation commerciale.
    \item Aucun tiers industriel ne peut revendiquer la propriété en l’absence d’une entente de recherche formelle.
\end{itemize}

Ainsi, le modèle de décision est juridiquement considéré comme une propriété institutionnelle, relevant de la mission académique de production et diffusion du savoir.

\subsubsection{Propriété des données collectées}
Les données de mobilité (images, LiDAR, localisation, comportements de piétons) collectées par NavetteAuto proviennent de l’espace public. 

Dans un contexte academique:

\begin{itemize}
    \item Elles sont considérées comme des données publiques appartenant à la municipalité ou à l’UQO si les capteurs opèrent sur un campus.
    \item Si elles contiennent des informations identifiables, elles sont régies par les normes éthiques universitaires, la Loi 25 et les protocoles institutionnels.
    \item L’équipe de recherche ne détient qu’un droit d’usage limité à la finalité explicite dans le protocole de recherche approuvé par le comité d’éthique.
\end{itemize}

En conséquence, les données n’appartiennent pas aux chercheurs, mais à l’entité publique responsable de l’espace dans lequel elles ont été collectées.

\subsubsection{Tension entre valorisation commerciale et souveraineté}
Dans un environnement universitaire, la finalité première de la collecte et de l’analyse des données est l’avancement des connaissances, la formation des étudiants et la production de résultats scientifiques reproductibles.

Cependant,ces données possèdent une valeur commerciale importante (entraînement de modèles, optimisation du transport urbain, partenariats industriels), ce qui crée une tension institutionnelle, comme l'indique Johnson dans \cite{johnson2023smartcities} :

\begin{itemize}
    \item L’UQO souhaite valoriser la recherche (brevets, licences, partenariats).
    \item La municipalité veut préserver la souveraineté des données publiques.
    \item Les citoyens exigent la protection de leur vie privée et un usage éthique.
\end{itemize}

Dans notre cadre académique, la souveraineté publique doit primer, et toute valorisation doit être encadrée par des conventions transparentes.

\subsubsection{Modèle de gouvernance recommandé des données}
L’importance d’un modèle de gouvernance transparent et centré sur la protection des citoyens dans les systèmes intelligents comme le soulève  \cite{Cen2024AIaudit},\cite{johnson2023smartcities},\cite{TransportCanada2024SafetyFramework} est nécessaire pour concilier les impératifs de recherche et la souveraineté publique  :

\begin{itemize}
    \item Propriété publique des données : la municipalité ou l’UQO demeure propriétaire.
    \item Droit d’usage académique encadré : les données sont utilisées exclusivement pour la recherche approuvée.
    \item Minimisation et anonymisation systématiques : seules les données nécessaires sont conservées.
    \item Transparence scientifique : publication des méthodologies et registre de traçabilité.
    \item Partenariats encadrés par des licences universitaires pour toute exploitation commerciale.
\end{itemize}

Ce modèle garantit un équilibre entre innovation, éthique et souveraineté.

\subsection{Documentation et responsabilité légale}

\subsubsection{Acteurs impliqués et responsabilité en cas d’accident}
Dans un cadre académique, et en l’absence de conducteur humain, le tableau 1, repartit la responsabilité d’un accident impliquant NavetteAuto entre plusieurs acteurs :

\begin{longtable}{|>{\raggedright\bfseries}p{4.5cm}|>{\raggedright}p{8.5cm}|}
\hline
Acteur & Responsabilité potentielle \\ \hline
Équipe de développement & Défaut de conception ou bug algorithmique. \\ \hline
UQO & Autorisation, supervision et infrastructure. \\ \hline
Opérateur technique & Maintenance, calibrations, supervision de sécurité. \\ \hline
Fabricant du matériel & Défaillance du LiDAR, freinage ou capteurs. \\ \hline
Municipalité ou campus & Infrastructure déficiente (signalisation, aménagement). \\ \hline
\end{longtable}

L’évaluation post-accident doit donc permettre d’identifier précisément la contribution de chaque acteur.

\subsubsection{Cadres juridiques applicables}
\begin{itemize}
    \item La responsabilité civile : faute, négligence, manquement aux obligations de sécurité.
    \item La responsabilité du fabricant : pour les composants défectueux.
    \item Obligations institutionnelles : conformité aux normes d’éthique et protocoles internes.
    \item Normes techniques internationales concernant la sécurité des systèmes autonomes.
\end{itemize}

\subsubsection{Importance de la traçabilité}
La transparence et la traçabilité constituent aussi des principes soulignés dans les analyses de la gestion des données dans les villes intelligentes \cite{Cen2024AIaudit}. Une traçabilité complète est essentielle pour :
\begin{itemize}
    \item Reconstituer la séquence d’événements.
    \item Analyser la perception et les décisions du système.
    \item Déterminer si un défaut algorithmique ou matériel est impliqué.
    \item Établir la responsabilité exacte de chaque acteur.
\end{itemize}

Dans un contexte académique, elle constitue également une exigence éthique pour garantir la reproductibilité scientifique.

\subsubsection{Plan de documentation proposé}
Un plan structuré de documentation est recommandé pour assurer l’auditabilité du système :

\begin{itemize}
    \item Journalisation technique : décisions de l’algorithme, erreurs ou défaillances, version du logiciel et modèles.
    \item Journalisation contextuelle : météo, visibilité, densité de piétons, vitesses, état de la route.
    \item Traçabilité logicielle : historique des entraînements, dates des mises à jour, paramètres et hyperparamètres.
    \item Registre d’interventions humaines : arrêts d’urgence, recalibrations, maintenance.
    \item Boîte noire sécurisée : enregistrement continu des 30–60 secondes avant et après l’accident, stockage chiffré, horodaté et inviolable.
\end{itemize}

\subsubsection{Processus d’audit post-accident}
Le protocole académique recommandé est le suivant :
\begin{enumerate}
    \item Gel des enregistrements et journaux.
    \item Extraction sécurisée par une équipe technique neutre.
    \item Analyse interne par l’équipe de recherche.
    \item Audit indépendant (services institutionnels, comité d’éthique).
    \item Rapport incluant causes probables, responsabilités identifiées, recommandations.
\end{enumerate}

Ce processus garantit transparence, rigueur scientifique et conformité éthique.