% -------------------------------
% 1. Classe et encodage
% -------------------------------
\documentclass[12pt]{article}

% -------- Packages généraux ---------
\usepackage[utf8]{inputenc}
\usepackage[T1]{fontenc}
\usepackage{lmodern}
\usepackage[french]{babel}
\usepackage{geometry}
\usepackage{setspace}
\usepackage{graphicx}
\usepackage{url}
\usepackage{csquotes}
\usepackage{caption}
\usepackage{subcaption}
\usepackage{enumitem}
\usepackage{array,longtable}
\usepackage{fancyhdr}
\usepackage{hyperref}  
\usepackage[backend=biber, style=ieee, sorting=nty, citestyle=numeric]{biblatex}

\addbibresource{references.bib}

% -------- Réglages ---------
\geometry{margin=1in}
\setlength{\parskip}{6pt}
\setlength{\parindent}{0pt}
\onehalfspacing

% Fancyhdr
\pagestyle{fancy}
\fancyhead[L]{Projet NavetteAuto}
\fancyhead[R]{UQO}
\fancyfoot[C]{\thepage}
\setlength{\headheight}{15pt}

% Hyperref pour liens noirs et sans encadré
\hypersetup{
    colorlinks=true,
    linkcolor=black,
    citecolor=black,
    urlcolor=black,
    pdfborder={0 0 0}
}

% Forcer les puces noires pour toutes les listes
\setlist[itemize]{label=\textbullet, left=0pt, itemsep=3pt}



\begin{document}


\tableofcontents
\newpage

% -------------------------------
% 5. Contenu principal
% -------------------------------

% Commencer la numérotation des sections à 6
\setcounter{section}{5}

\section{Gestion, Responsabilité et Propriété Intellectuelle}

\subsection{Propriété intellectuelle et souveraineté}

\subsubsection{Propriété du modèle de décision}
Dans un contexte académique, le modèle de décision développé dans le cadre du projet \textbf{NavetteAuto} incluant les mécanismes de planification de trajectoire et l’intégration logicielle constitue une création intellectuelle au sens de la législation sur les logiciels.

Dans une université, la règle générale est la suivante :

\begin{itemize}
    \item Les droits patrimoniaux appartiennent à l’UQO, conformément aux politiques internes sur la propriété intellectuelle des travaux réalisés dans le cadre de projets pédagogiques ou de recherche.
    \item Nous, les étudiants développeurs, conservons des droits moraux (paternité, intégrité de l’œuvre), mais n’avons pas la maîtrise de l’exploitation commerciale.
    \item Aucun tiers industriel ne peut revendiquer la propriété en l’absence d’une entente de recherche formelle.
\end{itemize}

Ainsi, le modèle de décision est juridiquement considéré comme une propriété institutionnelle, relevant de la mission académique de production et diffusion du savoir.

\subsubsection{Propriété des données collectées}
Les données de mobilité (images, LiDAR, localisation, comportements de piétons) collectées par NavetteAuto proviennent de l’espace public. 

Dans un contexte academique:

\begin{itemize}
    \item elles sont considérées comme des données publiques appartenant à la municipalité ou à l’établissement au cas où les capteurs opèrent sur un campus ;
    \item si elles contiennent des informations identifiables, elles sont régies par les normes éthiques universitaires, la Loi 25 et les protocoles institutionnels sur la protection des renseignements personnels ;
    \item l’équipe de recherche ne détient qu’un droit d’usage, limité à la finalité explicitée dans le protocole de recherche approuvé par le comité d’éthique.
\end{itemize}

En conséquence, les données n’appartiennent pas aux chercheurs, mais à l’entité publique responsable de l’espace dans lequel elles ont été collectées.

\subsubsection{Tension entre valorisation commerciale et souveraineté}
Dans un environnement universitaire, la finalité première de la collecte et de l’analyse des données est l’avancement des connaissances, la formation des étudiants et la production de résultats scientifiques reproductibles.

Cependant, ces mêmes données de mobilité possèdent une valeur commerciale importante (entraînement de modèles, optimisation du transport urbain, partenariats industriels), ce qui nourrit une tension institutionnelle, comme l'indique Johnson \cite{johnson2023smartcities}, l’innovation technologique dans les villes intelligentes doit impérativement être équilibrée avec la protection de la vie privée à savoir :

\begin{itemize}
    \item L’UQO souhaite valoriser la recherche (brevets, licences, partenariats).
    \item La municipalité veut préserver la souveraineté des données publiques.
    \item Les citoyens exigent la protection de leur vie privée et un usage éthique de leurs traces numériques.
\end{itemize}

Dans notre cadre académique, la souveraineté publique doit primer, et toute valorisation doit être encadrée par des conventions de recherche transparentes.

\subsubsection{Modèle de gouvernance recommandé des données}
L’importance d’un modèle de gouvernance transparent et centré sur la protection des citoyens dans les systèmes intelligents comme le soulève \cite{johnson2023smartcities},\cite{TransportCanada2024SafetyFramework},\cite{Cen2024AIaudit} est nécessaire pour concilier les impératifs de recherche et la souveraineté publique :


\begin{itemize}
    \item Propriété publique des données : La municipalité ou l’UQO demeure propriétaire des données brutes.
    \item Droit d’usage académique encadré : Les données sont utilisées exclusivement pour la recherche approuvée par le comité d’éthique.
    \item Minimisation et anonymisation systématiques : Conservation uniquement des données nécessaires à la recherche. Traitement préalable d’anonymisation avant tout entraînement de modèle.
    \item Transparence scientifique : Publication des méthodologies et limites du système.Mise en place d’un registre de traçabilité de la donnée utilisée pour l’entraînement.
    \item Partenariats encadrés par des licences universitaires : Toute exploitation commerciale doit passer par le service de valorisation de l’UQO.
\end{itemize}

Ce modèle garantit un équilibre entre innovation, éthique et souveraineté.

\subsection{Documentation et responsabilité légale}

\subsubsection{Acteurs impliqués et responsabilité en cas d’accident}

Dans un cadre académique, et en l’absence de conducteur humain, le tableau 1, repartit la responsabilité d’un accident impliquant NavetteAuto entre plusieurs acteurs :

\begin{longtable}{|>{\bfseries}m{5cm}|m{9cm}|}
\hline
Acteur & Responsabilité potentielle \\ \hline
Équipe de développement & Défaut de conception ou bug algorithmique. \\ \hline
UQO & Autorisation, supervision et infrastructure. \\ \hline
Opérateur technique & Maintenance, calibrations, supervision de sécurité. \\ \hline
Fabricant du matériel & Défaillance du LiDAR, freinage ou capteurs. \\ \hline
Municipalité / campus & Infrastructure déficiente (signalisation, aménagement). \\ \hline
\end{longtable}

L’évaluation post-accident doit donc permettre d’identifier précisément la contribution de chaque acteur.

\subsubsection{Cadres juridiques applicables}

Les principaux cadres juridiques pertinents sont :

\begin{itemize}
    \item La responsabilité civile : faute, négligence, manquement aux obligations de sécurité.
    \item La responsabilité du fabricant : pour les composants défectueux.
    \item Les obligations institutionnelles : conformité aux normes d’éthique, aux autorisations et aux protocoles internes.
    \item Les normes techniques internationales concernant la sécurité des systèmes autonomes.
\end{itemize}

\subsubsection{Importance de la traçabilité}

La transparence et la traçabilité constituent aussi des principes soulignés dans les analyses de la gestion des données dans les villes intelligentes \cite{johnson2023smartcities}. Une traçabilité complète est essentielle pour :

\begin{itemize}
    \item Reconstituer la séquence d’événements.
    \item Analyser la perception et les décisions du système.
    \item déterminer si un défaut algorithmique ou matériel est impliqué,
    \item établir la responsabilité exacte de chaque acteur.
\end{itemize}

Dans un contexte académique, elle constitue également une exigence éthique pour garantir la reproductibilité scientifique.

\subsubsection{Plan de documentation proposé}
Un plan structuré de documentation est recommandé pour assurer l’auditabilité du système :

\begin{itemize}
    \item Journalisation technique : décisions de l’algorithme, erreurs ou défaillances capteurs, version du logiciel et des modèles entraînés.
    \item Journalisation contextuelle : météo, visibilité, densité de piétons, vitesses, état de la route.
    \item Traçabilité logicielle : historique des entraînements, dates des mises à jour, paramètres et hyperparamètres utilisés
    \item Registre d’interventions humaines : arrêts d’urgence, recalibrations, opérations de maintenance.
    \item Boîte noire sécurisée : enregistrement continu des 30–60 secondes avant et après l’accident ; stockage chiffré, horodaté et inviolable.
\end{itemize}

\subsubsection{Processus d’audit post-accident}
Le protocole académique recommandé est le suivant :

\begin{enumerate}
    \item Gel des enregistrements et des journaux.
    \item Extraction sécurisée des données par une équipe technique neutre.
    \item Analyse interne par l’équipe de recherche.
    \item Audit indépendant (services institutionnels, comité d’éthique).
    \item Rédaction d’un rapport incluant : causes probables, responsabilités identifiées, recommandations pour les futures expérimentations.
\end{enumerate}

Ce processus garantit la transparence, la rigueur scientifique et la conformité éthique.

% -------------------------------
% 6. Bibliographie
% -------------------------------
\printbibliography

\end{document}
